\section{Methods}

\subsection{Dataset}
This consists of images, their descriptions and the textual context in which these images occur. Many datasets, such as ImageNET \cite{Russakovsky2012} and MSCOCO \cite{Lin2014}, exist of which many only consist of images and their description. Though, next to images and their annotations, in this approach the dataset also needs to include the context in which the image is in. In the literature the ImageCLEF dataset is used for various image retrieval tasks as well as for image annotation tasks. In order to compare the models in the literature with the model in the thesis and since this dataset contains the required components to develop and train a model for automatic image description generation, the ImageCLEF 2014 dataset is a suitable candidate. Furthermore, this dataset also contains the webpages on which the images are hosted. This can be used as the context for the images. However, the model of this thesis will be used by Dedicon and data that they have might not represent the data that is used by ImageCLEF. First of all the language (English vs. Dutch) is different which will have a big impact if this model were to be trained on the ImageCLEF dataset and then tested on the Dedicon dataset. Furthermore, the images in the Dedicon dataset might be of an entirely different category compared with the images in the ImageCLEF dataset. Therefore, only one of the datasets can be used to both train and test the model on. Once a working prototype of the model has been created and trained, the model is tested on a subset of the dataset that is used. 
